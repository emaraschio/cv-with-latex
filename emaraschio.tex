%!TEX TS-program = xelatex
%!TEX encoding = UTF-8 Unicode
% A4 paper size by default, use 'letterpaper' for US letter
\documentclass[11pt, a4paper]{awesome-cv}

% Configure page margins with geometry
\geometry{left=1.4cm, top=.8cm, right=1.4cm, bottom=1.8cm, footskip=.5cm}

% Specify the location of the included fonts
\fontdir[fonts/]

% Color for highlights
% Awesome Colors: awesome-emerald, awesome-skyblue, awesome-red, awesome-pink, awesome-orange
%                 awesome-nephritis, awesome-concrete, awesome-darknight
\colorlet{awesome}{awesome-darknight}
% Uncomment if you would like to specify your own color
% \definecolor{awesome}{HTML}{CA63A8}

% Colors for text
% Uncomment if you would like to specify your own color
%\definecolor{darktext}{HTML}{414141}
% \definecolor{text}{HTML}{333333}
\definecolor{graytext}{HTML}{5D5D5D}
% \definecolor{lighttext}{HTML}{999999}

% Set false if you don't want to highlight section with awesome color
\setbool{acvSectionColorHighlight}{false}

% If you would like to change the social information separator from a pipe (|) to something else
\renewcommand{\acvHeaderSocialSep}{\quad\textbar\quad}

% Available options: circle|rectangle,edge/noedge,left/right
% \photo[rectangle,edge,right]{./examples/profile}
\name{Ezequiel}{Maraschio}
\position{Senior Software Engineer}
\address{Buenos Aires, Argentina}

\mobile{(+54) 911-6042-3765}
\email{e@maraschio.com}
\homepage{maraschio.com}
\github{emaraschio}
\linkedin{emaraschio}
\twitter{emaraschio}
% \skype{ezequiel.maraschio}
% \extrainfo{extra informations}

\quote
	{``A mind is like a parachute. It doesn't work if it is not open." - Frank Zappa}

\begin{document}

% Print the header with above personal informations
% Give optional argument to change alignment(C: center, L: left, R: right)
\makecvheader[C]

% Print the footer with 3 arguments(<left>, <center>, <right>)
% Leave any of these blank if they are not needed
\makecvfooter
  {}
  {Ezequiel Maraschio~~~·~~~Resume}
  {}

\cvsection{Summary}

\begin{cvparagraph}

With 18+ years of experience, I've evolved from a passionate Senior Software Engineer to a results-driven Engineering Manager. My expertise spans building complex web applications and leading high-performing teams. I thrive on creating impactful solutions, mentoring engineers, and driving strategic projects.

My hands-on approach ensures technical excellence and fosters a collaborative environment, delivering significant value to the organization.

\end{cvparagraph}

\cvsection{Work Experience}

\begin{cventries}

  \cventry
    {Senior Software Engineer}
    {Thirty Madison}
    {Remote, USA}
    {May 2018 - Present}
    {
      \begin{cvitems}
        \item {Collaborated with a team of +30 engineers to define, design and develop a Health-care platform and several SPAs for Customers, Cx Agents and Doctors}
        \item {Tech Stack: Ruby on Rails, React, TypeScript, GraphQL, Cypress, PostgreSQL, Redis, AWS (EC2, S3, RDS, Fargate, CloudFront), HTML \& CSS}
      \end{cvitems}
    }

  \cventry
    {Senior Software Engineer}
    {Manas Tech}
    {Remote, Argentina}
    {Dec 2017 - May 2018}
    {
      \begin{cvitems}
        \item {Collaborated with a team of 4 engineers to develop a CMS for an Audit Company and a Funnel to create monitoring campaigns of epidemic diseases}
        \item {Played a key role in implementing Agile practices \& Training on both projects and across the company}
         \item {Tech Stack: Ruby on Rails, React, Redux, Vue, Elixir, Phoenix, Crystal, GraphQL, Cypress IO, PostgreSQL, Redis, Docker, Rancher, AWS (EC2, S3, RDS, Route 53, ELB), HTML \& CSS}
      \end{cvitems}
    }

  \cventry
    {Senior Software Engineer}
    {Deviget LLC}
    {Remote, USA / Argentina}
    {Feb 2012 - Sep 2017}
    {
      \begin{cvitems}
        \item {Collaborated with several clients and teams to implement APIs for internal and external usage and played a key role on develop MVPs for new clients}
        \item {Played a key role in Agile Coaching \& Training on several projects, from Scrum adoption to level-up Teams}
        \item {Tech Stack: Ruby, Ruby on Rails, Node.js, Angular, React, PHP, Laravel, MySQL, MongoDB, ElasticSearch, PostgreSQL, Redis, Selenium, Heroku, AWS (EC2, S3, RDS, CloudFront, Route 53, ElastiCache, ELB), HTML \& CSS}
      \end{cvitems}
    }

  \cventry
    {Co-Founder \& Software Engineer}
    {Zarego}
    {CABA, Argentina}
    {Jul 2012 - Oct 2014}
    {
      \begin{cvitems}
        \item {Played a key role as Pre-sales engineer and implemented MVPs for Advertising and marketing industries}
        \item {Tech Stack: Ruby on Rails, Node.js, Angular, PHP, Symfony, MySQL, MongoDB, ElasticSearch \& AWS (EC2, S3, RDS, CloudFront, ELB)}
      \end{cvitems}
    }

  \cventry
    {Software Engineer}
    {TeraCode SA}
    {CABA, Argentina}
    {May 2011 - Feb 2012}
    {
      \begin{cvitems}
        \item {Collaborated with a team of 8 engineers to develop a Funnel Widget to sold solar panels within partner\'s sites}
        \item {Tech Stack: Ruby on Rails, PHP, Symfony, JS MVC, Backbone, MySQL, HTML \& CSS}
      \end{cvitems}
    }

  \cventry
    {Software Engineer \& Team Lead}
    {E-planning SA}
    {CABA, Argentina}
    {Apr 2010 - May 2011}
    {
      \begin{cvitems}
        \item {Implemented application logic and wrote API services for a high-availability Ad-Server solution}
        \item {Played a key role in Agile Adoption \& Training within Scrum Framework and owned the Agile process across the company}
        \item {Tech Stack: PHP, Perl, JS, Selenium, MySQL, HTML \& CSS}
      \end{cvitems}
    }

  \cventry
    {Software Engineer}
    {Avatar SA}
    {CABA, Argentina}
    {Mar 2008 - Mar 2010}
    {
      \begin{cvitems}
        \item {Implemented application logic and wrote API services behind several S\&P 500 companies websites}
        \item {Played a key role in Agile Adoption within Scrum Framework}
        \item {Tech Stack: PHP, C\#, JS, ASP.NET, Symfony, Selenium, MSSQL, PostgreSQL, MySQL, HTML \& CSS}
      \end{cvitems}
    }

  \cventry
    {Jr. Software Engineer}
    {Infocomercial SH}
    {CABA, Argentina}
    {Mar 2007 - Mar 2008}
    {
      \begin{cvitems}
        \item {Implemented application logic and layouts for custom CMS for external clients}
        \item {I owned all the technical support \& assistance for clients}
        \item {Tech Stack: PHP, MySQL, JS, HTML \& CSS}
      \end{cvitems}
    }

\end{cventries}

% \cvsection{Public Speaking}

\begin{cventries}
  \cventry
    {Full Stack Tech Conf: https://ar-fullstack.tech/}
    {Speaker for <Full Stack Tech Conf AR 2nd Edition>}
    {CABA, Argentina}
    {Nov 2017}
    {
      \begin{cvitems}
        \item {Workshop Fullstack End2End - Extended version: https://github.com/emaraschio/workshop-fullstack-conf-2017}
      \end{cvitems}
    }

%  \cventry
%    {EXO Training: http://www.exotraining.com/ }
%    {Speaker for <Passion for Tech Meetup>}
%    {CABA, Argentina}
%    {Aug 2017}
%    {
%      \begin{cvitems}
%        \item {Let's talk about Web Browsers - Extended version}
%      \end{cvitems}
%    }

%   \cventry
%     {SysARmy Community Group - Support for those who give support: https://sysarmy.com.ar/ }
%     {Speaker for <MeetARmy Meetup>}
%     {CABA, Argentina}
%     {Mar 2017}
%     {
%       \begin{cvitems}
%         \item {TODO List - Two-edged sword}
%       \end{cvitems}
%     }

  \cventry
    {Go Lang AR Community Group: https://golang.com.ar/ }
    {Speaker for <Go Lang AR 2nd Meetup>}
    {CABA, Argentina}
    {Feb 2017}
    {
      \begin{cvitems}
        \item {Graphite with Go-Carbon}
      \end{cvitems}
    }

  \cventry
    {Rubysur Community Group: http://rubysur.org/}
    {Speaker for <Rubysur Meetup>}
    {CABA, Argentina}
    {July 2016}
    {
      \begin{cvitems}
        \item {Productivity Tips}
      \end{cvitems}
    }

  \cventry
    {Nerdearla - For Nerds by Nerds: http://nerdear.la/}
    {Speaker for <Nerdearla 3rd Edition>}
    {CABA, Argentina}
    {June 2016}
    {
      \begin{cvitems}
        \item {How Web browsers works}
      \end{cvitems}
    }

  \cventry
    {Nerdearla - For Nerds by Nerds: http://nerdear.la/}
    {Speaker for <Nerdearla 2nd Edition>}
    {CABA, Argentina}
    {June 2015}
    {
      \begin{cvitems}
        \item {Software Development Anti-Patterns}
      \end{cvitems}
    }

  \cventry
    {Nerdearla - For Nerds by Nerds: http://nerdear.la/}
    {Speaker for <Nerdearla 1st Edition>}
    {CABA, Argentina}
    {March 2014}
    {
      \begin{cvitems}
        \item {Personal Organization Framework}
      \end{cvitems}
    }

  \cventry
    {GULBAC Conf: https://gulbac.org/}
    {Speaker for GULBAC Conf 1st Edition>}
    {Mar del Plata, Argentina}
    {June 2013}
    {
      \begin{cvitems}
        \item {Personal Organization Framework}
        \item {Scrum Inception}
      \end{cvitems}
    }
\end{cventries}

% \cvsection{Education}

\begin{cventries}
  \cventry
    {Agile Team Facilitator (ATF)} % Degree
    {Kleer} % Institution
    {CABA, Argentina} % Location
    {Nov 2017} % Date(s)
    {}

  \cventry
    {Certified Scrum Product Owner (CSPO)} % Degree
    {Scrum Alliance} % Institution
    {CABA, Argentina} % Location
    {Sep 2013} % Date(s)
    {}

  \cventry
    {Certified Scrum Master (CSM)} % Degree
    {Scrum Alliance} % Institution
    {CABA, Argentina} % Location
    {Oct 2011} % Date(s)
    {}

  \cventry
    {B.S. in Computer Science and Engineering} % Degree
    {UAI} % Institution
    {CABA, Argentina} % Location
    {Mar 2006 - Mar 2013} % Date(s)
    {Unfinished}

\end{cventries}

% \cvsection{Languages}

\begin{cventries}

  \cventry
    {English} % Degree
    {Professional working proficiency}
    {}
    {}
    {}

  \cventry
    {Spanish} % Degree
    {Native or bilingual proficiency}
    {}
    {}
    {}
    
\end{cventries}


\end{document}
